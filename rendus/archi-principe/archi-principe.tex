\documentclass[headings=standardclasses,parskip=half]{scrartcl}

\usepackage[french]{babel}
\usepackage[margin=3cm]{geometry}
\usepackage{graphicx}
\usepackage[hidelinks]{hyperref}

\titlehead{
    \begin{center}
        \includegraphics[width=5cm]{n7.png}
    \end{center}
}
\subject{Projet Données Réparties\\Sciences du Numérique – Semestre 7}
\title{Architecture de principe}
\subtitle{}
\author{Enzo PETIT \and Nam VU}
\date{29 novembre 2021}
\publishers{}


\begin{document}

\maketitle

% 1. Un plan de travail initial par groupe.
% Ce plan de travail doit lister les tâches à réaliser pour le groupe,
% et indiquer pour chacune des tâches, la personne affectée à la tâche
% 
% 2. L'architecture de principe de la plateforme Linda à réaliser. 
% Ce document ne devrait pas excéder 1 à 2 page(s) A4. 
% Il devrait comporter
% * les principales classes envisagées
% * les difficultés identifiées
% * au besoin, des diagrammes de séquence à la UML, pour détailler les protocoles un peu complexes.
% * le type et l'organisation des tests envisagés

\section*{Plan de travail initial}

\begin{enumerate}
    \item \texttt{CentralizedLinda} sans gestion des callbacks
          \begin{enumerate}
              \item Savegarde des tuples en mémoire (\texttt{write}) \emph{— Nam}
              \item Méthodes non bloquantes 
              (\texttt{tryTake}, \texttt{tryRead}, \texttt{takeAll}, \texttt{readAll}) \emph{— Enzo}
              \item Méthodes bloquantes (\texttt{take}, \texttt{read}) \emph{— Nam}
          \end{enumerate}
    \item \texttt{CentralizedLinda} avec callbacks (\texttt{eventRegister}) \emph{— Enzo}
    \item Classes client/mono-serveur sans callbacks \emph{— Enzo}
    \item Classes client/mono-serveur avec callbacks \emph{— Nam}
\end{enumerate}

L'écriture de tests se fera au fur et à mesure de manière croisée par chacun d'entre nous
(e.g. Nam écrira les tests sur les méthodes non bloquantes qu'Enzo a implémenté).

\section*{Architecture de principe}

\subsection*{Classes envisagées}

Pour la version mémoire partagée à priori rien de plus que la classe \texttt{CentralizedLinda}.

Pour la version client/mono-serveur on aura les classes \texttt{LindaClient},
\texttt{LindaServer} (interface du serveur) et \texttt{LindaServerImpl} (implémentation du serveur).

\subsection*{Difficultés identifiées}

Quelle structure de donnée Java pour sauvegarder les tuples reçus ?

\subsubsection*{Organisation des tests}

Tests unitaires sous JUnit (ou sans si ça s'avère trop compliqué/long à mettre en place).

\end{document}
